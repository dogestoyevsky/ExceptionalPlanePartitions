\documentclass[10pt,letter]{article}
\usepackage{amsmath,amssymb,graphicx,setspace,fullpage,breqn}
\onehalfspacing
\usepackage{fullpage}
\DeclareMathOperator*{\argmin}{arg\,min}

\begin{document}

\title{Differential Geometry Midterm}
\author{Holly Mandel}
\maketitle 

\paragraph*{1.} \textbf{$\mathbb{C}P^n$ is a smooth complex manifold.} $\mathbb{C}P^n$ is defined as the quotient space $\mathbb{C}^{n+1} \setminus \lbrace 0 \rbrace/\sim$, where $x \sim \lambda x$ for all $\lambda \in \mathbb{C}^\times$. Let $p$ be the projection map. 
\begin{itemize}
\item $\mathbb{C}P^n$ is Hausdorff. For the quotient $X / \sim$ of a Hausdorff space $X$ is Hausdorff exactly when the set $D = (x_1,x_2): x_1 \sim x_2$ is a closed subset of $X \times X$. But $D = \lbrace(z_0,...,z_n,\lambda z_0,...,\lambda z_n ) : (z_0,...,z_n) \in \mathbb{C}^{n+1}, \lambda \in \mathbb{C} \rbrace \subset \mathbb{C}^{2n+2}$ is the intersection of the zero loci of the functions $P_{ij}: \mathbb{C}^{2n+2} \rightarrow \mathbb{C}$:  
\begin{equation*}
P_{ij}(z_0,...,z_n,w_0,...,w_n) =  z_iw_j - w_iz_j 
\end{equation*}
for $0 \leq i < j \leq n$. Therefore, $D$ is closed, so $\mathbb{C}P^n$ is Hausdorff. 

\item $\mathbb{C}P^n$ is second-countable. For if $f: X \rightarrow Y$ is an open surjection and $X$ is second countable, $Y$ is second countable. The quotient map $p$ is surjective by definition. We claim that $p$ is an open map. For if $\mathcal{O}$ is an open set in $\mathbb{C}^{n+1}$, say $x \in p^{-1}(p(\mathcal{O}))$. Then there exists a nonzero $\lambda$ such that $y = \lambda x \in \mathcal{O}$. By the openness of $\mathcal{O}$, there is some neighborhood $N_{\eta}(y) \subset \mathcal{O}$. But the function $x \mapsto \lambda \cdot x$ is a continuous function, so for sufficiently small $\epsilon$, $\lambda \cdot N_{\epsilon}(x) \subset N_{\eta}(y) \subset \mathcal{O}$. Thus $p^{-1}(p(\mathcal{O}))$ is open, so by definition $p(\mathcal{O})$ is open.  Therefore, $p$ is an open map, and $\mathbb{C}P^n$ is second-countable. 
\end{itemize}


An element of $\mathbb{C}P^n$ can be represented as $[(z_0,z_1,...,z_n)]$, where $(z_0,...,z_n) \in \mathbb{C}^{n+1}$ is a point on the line represented. To construct charts on $\mathbb{C}P^n$, Let $U_i = \lbrace [(z_0,z_1,...,z_n)] \in \mathbb{C}P^n: z_i \neq 0 \rbrace$ for $i = 0,...,n$. Though an element of $\mathbb{C}P^n$ has many representatives, they will either all be zero in one coordinate or all not be zero in that coordinate, so this set is well-defined. Define the map $\phi_i: U_i \rightarrow \mathbb{C}^n$ as follows:
\begin{equation*}
\phi_i([z_0,...,z_n]) = (\frac{z_0}{z_i},...,\frac{z_{i-1}}{z_i},\frac{z_i+1}{z_i},...,\frac{z_n}{z_i})
\end{equation*}
Then $\phi_i$ is well defined on $U_i$ because $z_i$ is nonzero for any representative of $[z_0,...,z_n]$, and all representatives map to the same point in $\mathbb{C}^n$. Since every element of $\mathbb{C}P^n$ is represented by points with at least one nonzero coordinate, the $U_i$ cover $\mathbb{C}P^n$. The $U_i$ are open, because the preimage of $U_i$ under the quotient map is the open set $\mathbb{C}^{n+1} \setminus \lbrace z_i = 0 \rbrace$. Therefore, the collection $\lbrace (U_i,\phi_i): i = 0,...,n \rbrace$ give an atlas for $\mathbb{C}P^n$.

Finally, we verify that transition maps given by these charts are analytic. Take $(w_1,...,w_n) \in \phi_i(U_i \cap U_j)$. Assume first that $i > j$. Then
\begin{dmath*}
\phi_j \circ \phi_i^{-1}(w_1,...,w_n) = \phi_j([w_1,...,w_{i-1},1,w_{i+1},...,w_n]) = (\frac{w_1}{w_{j+1}},...,\frac{w_{j}}{w_{j+1}},\frac{w_{j+2}}{w_{j+1}},...,\frac{w_{i-1}}{w_{j+1}},\frac{1}{w_{j+1}},\frac{w_{i+1}}{w_{j+1}},...,\frac{w_{n}}{w_{j+1}})
\end{dmath*}
In this case, $w_{j+1}$ is nonzero because it occupies that $j^{th}$ position of $0,1,...,n$ in $\mathbb{C}^{n+1}$, and $\phi^{-1}(w_1,...,w_n) \in U_j$. On the other hand, if $j > i$, then 
\begin{dmath*}
\phi_j \circ \phi_i^{-1}(w_1,...,w_n) = \phi_j([w_1,...,w_{i-1},1,w_{i+1},...,w_n]) = (\frac{w_1}{w_{j}},...,\frac{w_{i-1}}{w_{j}},\frac{1}{w_{j}},\frac{w_{i}}{w_{j}},...,\frac{w_{j-1}}{w_j},\frac{w_{j+1}}{w_j},...,\frac{w_{n}}{w_{j}})
\end{dmath*}
In this case, $w_j$ is nonzero because it occupies the $j^{th}$ position in $\mathbb{C}^{n+1}$. Therefore, in both cases the component functions of each of these maps are analytic in each variable, since they are quotients whose denominators do not vanish on the domain. This confirms that $\mathbb{C}P^n$ is a complex manifold.

\paragraph*{2.} \textbf{Isometry between the plane and the punctured sphere. } 
\subparagraph*{(a.)} By definition, the arc length of a curve $\gamma: I \rightarrow (M,g)$ is
\begin{equation*}
L(\gamma) = \int_{0}^1 \sqrt{g_{\gamma(t)}(\gamma'(t),\gamma'(t))} \, dt
\end{equation*}
In our example $(\mathbb{R}^n,\hat{g})$, we have $\gamma(t) = (t,0,...,0)$ with velocity vector $\gamma'(t) = dx_1$. We compute
\begin{dmath*}
L(\gamma) = \int_{0}^1 \sqrt{\frac{4}{(1+t^2)^2}} \, dt 
=  \int_{0}^1 \frac{2}{1+t^2} \, dt
= 2 \cdot \int_{\arctan(0)}^{\arctan(1)} d \theta
= \pi/2
\end{dmath*}

\subparagraph*{(b.)} As discussed in class, the orthogonal group $O(3)$ is a transitive group is isometries of $S^2 \subset \mathbb{R}^3$. Therefore if $N = (0,0,1)$ is the north pole of the sphere, then there is $A \in O(3)$ such that $Ap = N$, and $A: S^2 \setminus \lbrace p \rbrace \rightarrow S^2 \setminus \lbrace N \rbrace$ is an isometry. Since isometry is a transitive relation, we therefore may assume that $p = N$. 

We have the stereographic projection $\phi: S^2 \setminus \lbrace N \rbrace \rightarrow \mathbb{R}^2$:
\begin{equation*}
\phi(x_1,x_2,x_3) = \frac{1}{1-x_1}(x_2,x_3)
\end{equation*}
We can compute the inverse $\phi^{-1}(y_1,y_2)$. For since $\vert x \vert^2 = 1$, we have the relation $y_1^2 + y_2^2 = \frac{1+x_1}{1-x_1}$. This implies that $x_1 =\frac{ \Vert y \Vert^2 - 1}{\Vert y \Vert^2 + 1}$. Then $x_i = \frac{2 y_i}{\Vert y \Vert^2 + 1}$ for $i = 2,3$. Therefore 
\begin{equation*}
\phi^{-1}(y_1,y_2) = \bigg(1-\frac{2}{(y_1^2 + y_2^2) + 1},\frac{2y_1}{(y_1^2 + y_2^2) + 1},\frac{2y_2}{(y_1^2 + y_2^2) + 1} \bigg)
\end{equation*}

Now we compute the differential of $\phi^{-1}(y_1,y_2)$:
\begin{equation*}
D(\phi^{-1})_{(y_1,y_2)} = 
\left(
\begin{array}{cc}
\frac{4y_1}{((y_1^2 + y_2^2) + 1)^2} & \frac{4y_2}{((y_1^2 + y_2^2) + 1)^2}  \\ 
\frac{2}{(y_1^2 + y_2^2) + 1} + \frac{-4y_1^2}{((y_1^2 + y_2^2) + 1)^2} &  \frac{-4y_1y_2}{((y_1^2 + y_2^2) + 1)^2} \\
\frac{-4y_1y_2}{((y_1^2 + y_2^2) + 1)^2} & \frac{2}{(y_1^2 + y_2^2) + 1} + \frac{-4y_2^2}{((y_1^2 + y_2^2) + 1)^2} 
\end{array}
\right)
\end{equation*}

Let $\delta_1$ and $\delta_2$ be the partial differential operators on $\mathbb{R}^2$ than span $T_y\mathbb{R}^2$ for all $y \in \mathbb{R}^2$. Then $D(\phi^{-1})_y \delta_1$ and $D(\phi^{-1})_y \delta_2$ form a basis for $T_{x}\mathbb{S}^2$, where $x = \phi^{-1}(y)$. Let $\langle \cdot , \cdot \rangle_x$ be the induced metric from $\mathbb{R}^3$ on $S^2$. Then 
\begin{center}
$\langle D(\phi^{-1})_y \delta_1,D(\phi^{-1})_y \delta_2 \rangle_x = 0$ \\
$\langle D(\phi^{-1})_y \delta_1,D(\phi^{-1})_y \delta_1 \rangle_x = \langle D(\phi^{-1})_y \delta_2,D(\phi^{-1})_y \delta_2 \rangle_x = \frac{4}{(1+\Vert y \Vert^2)^2}$
\end{center} 
(See calculations attached). Therefore the map $\phi^{-1}: (\mathbb{R}^2,\hat{g}) \rightarrow (S^2,g_{S^n})$ is conformal. For given an arbitrary vector $v = v_1 (\delta_1)_p + v_2 (\delta_2)p$, on the one hand
\begin{dmath*}
\hat{g}_p(v,v) = v_1^2 \hat{g}_p((\delta_1)_p,(\delta_1)_p) + v_2^2 \hat{g}_p((\delta_2)_p,(\delta_2)_p) + 2v_1 v_2 \hat{g}_p((\delta_1)_p,(\delta_2)_p)
=  \frac{4 \cdot (v_1^2 + v_2^2)}{1+\vert p \vert^2}
\end{dmath*}
and on the other hand 
\begin{dmath*}
(g_{S^n})_{\phi^{-1}(p)}(D\phi^{-1}_p(v),D\phi^{-1}_p(v)) = v_1^2 \langle D(\phi^{-1})_p \delta_1,D(\phi^{-1})_p \delta_1 \rangle_p + v_1^2 \langle D(\phi^{-1})_p \delta_2,D(\phi^{-1})_p \delta_2 \rangle_p + 2v_1 v_2 \langle D(\phi^{-1})_p \delta_1,D(\phi^{-1})_p \delta_2 \rangle_p
= \frac{4 \cdot (v_1^2 + v_2^2)}{1+\vert p \vert^2}
\end{dmath*}
By the polarization identity, this proves that $\hat{g}_p(X_p,Y_p) = (g_{S^n})_p(D(\phi^{-1})_p X_p,D(\phi^{-1})_p Y_p)$ for all $p \in \mathbb{R}^2$ and all $X_p,Y_p \in T_p\mathbb{R}^2$. Therefore $\phi^{-1}$ gives a conformal map between these two Riemannian manifolds with conformal factor $1 = e^0$.
 
\paragraph*{3.} \textbf{Normal bundle is a smooth vector bundle.} The normal space to a submanifold $M^m$ of a Riemannian manifold $(N^h,h)$ at a point $p$ is  
\begin{equation*}
N_pM = \lbrace v_p \in T_pM:  h(v_p,w_p) = 0, \ \forall w_p \in T_pM \rbrace
\end{equation*}
where $T_pM$ is the image of the tangent space to $M$ at $p$ under the differential of the inclusion map. This differential is injective, since its coordinate representation with respect to a slice chart (Gundmundson Definition 2.8) is the upper $m \times m$ identity matrix. The normal bundle of $M$ in $N$ is defined as 
\begin{equation*}
NM = \lbrace (p,v_p): p \in M, v_p \in N_p M \rbrace
\end{equation*}


Let $\pi: NM \rightarrow M$ be the projection map $(p,v_p) \mapsto p$. By the definiton of $NM$, $\pi$ is surjective. In addition, $\pi^{-1}(\lbrace p \rbrace) = \lbrace (q,w) \in NM: \pi(q,w) = p \rbrace = \lbrace p \rbrace \times N_pM$. But $N_pM \simeq \mathbb{R}^{n-m}$, since $T_pM$ has dimension $m$ at every point, and the differential of the inclusion map is injective, so its orthogonal complement in $T_pN$ has dimension $n-m$. This confirms Criterion (i) in Gundmundsson Def. 4.1.

By Gundmundsson Definition 2.8, for any $p \in M \subset N$ there is a chart $(U,x)$ for $U \subset N$ such that $x(U \cap M) = x(U) \cap (\mathbb{R}^{m} \times \lbrace 0 \rbrace^{n-m})$.  Given such a chart, let $X_i = D(x^{-1})_{x(p)}(\delta_i)$ be the pushforward of the $i^{th}$ partial differential operator on $\mathbb{R}^n$, giving a basis of $T_pN$. 

We produce a set of pointwise-orthonormal vector fields $Y^1,...,Y^m$ with the property that $\text{Span}(Y^1_p,...,Y^k_p) = \text{Span}(X^1_p,...,X^k_p)$ for every $p \in U$ and $k \leq m$. First, set 
\begin{equation*}
Y^1_p = \frac{X^1_p}{\sqrt{h_p(X^1_p,X^1_p)}}
\end{equation*}
The denominator is nonvanishing because $h_p$ is positive definite and $Dx^{-1}$ is nonsingular on $x(U)$. By the smoothness of $h$ and $X^1$, $Y^1$ is a smooth vector field. Clearly the span of $X^1_p$ is the span of $Y^1_p$. 

For the inductive step, say we are given $Y_1,...,Y_k$ as described. Define $\hat{Y}^{k+1}$ inductively as
\begin{equation*}
\hat{Y}^{k+1}_p = X^{k+1}_p - \sum_{i=1}^k h_p(X^{k+1}_p,Y^k_p) \, Y^k_p
\end{equation*}
Then $\hat{Y}^{k+1}$ has nonzero magnitude everywhere, because for all $p$, $X^{k+1}_p \not \in \text{Span}(X^1_p,...,X^k_p) = \text{Span}(Y^1_p,...,Y^k_p)$. Therefore define 
\begin{equation*}
Y^{k+1}_p = \frac{\hat{Y}^{k+1}_p}{\sqrt{h_p(\hat{Y}^{k+1}_p,\hat{Y}^{k+1}_p)}}
\end{equation*}
It is clear that $Y^{k+1}_p$ is orthogonal to $Y^j_p$ for $j \leq k$. But all $Y^j$ for $j \leq k+1$ are linear combinations of $X^1,...,X^{k+1}$, so we must have that $\text{Span}(Y^1_p,...,Y^{k+1}_p) = \text{Span}(X^1_p,...,X^{k+1}_p)$. This process stops when $k = m$, completing the proof. 

By our original choice of $x$, we now have that $Y^{m+1}_p,...,Y^{n}_p$ form a basis for $N_pM$. For $Y^1_p,...,Y^m_p$ have the same span as $X^1_p,...,X^m_p$ by construction, and the latter set is a basis for $T_pM$. The remaining $\lbrace Y^j_p \rbrace$ are orthogonal to these vectors, and therefore to $T_pM$, and span a space of dimension $n-m$, so they span $N_pM$. 

We can use this result to define a system of bundle charts on $NM$. For since $\lbrace Y^i_p \rbrace$ form a basis of $T_pN$ for every $p$, every point in $\pi^{-1}(U)$ can be written uniquely as $(p,\sum_{i=m+1}^{n+m} a_i Y^i_p)$. Therefore we can define $\psi: \pi^{-1}(U) \rightarrow U \times \mathbb{R}^{n-m}$ so that $(p,\sum_{i=m+1}^{n+m} a_i Y^i_p) \mapsto (p,(a_{m+1},...,a_{m+n}))$. Because we can choose such a $(U,x)$ containing every point in $M$, the $\pi^{-1}(U)$ cover $NM$. 

We show that transition maps in this atlas are smooth. Say that two charts $(\pi^{-1}(U_1),\psi_1)$ and $(\pi^{-1}(U_2),\psi_2)$ such that $U_1 \cap U_2 \neq \emptyset$. For any $p \in U_1 \cap U_2$, choose a chart $(U,x)$ of $N$ such that $p \in U$ and let $V = U_1 \cap U_2 \cap U$. $(U,x)$ induces a canonical chart on $TN$. Let $\lbrace Q^i \rbrace$ be the orthonormal vectors corresponding to $\psi_1$ and $\lbrace R^i \rbrace$ be the orthonormal vectors corresponding to $\psi_2$. The vector fields $Q^i$ and $R^i$ have a smooth coordinate representatiosn with respect to $(U,x)$:
\begin{eqnarray*}
Q^i(p) = \sum_{j=1}^n a_{ij}(p) \frac{d}{dx_i}\bigg\vert_p \\
R^i(p) = \sum_{j=1}^n b_{ij}(p) \frac{d}{dx_i}\bigg\vert_p
\end{eqnarray*}
where $\frac{d}{dx_i}\vert_p$ are the pushforwards of the differential operator on $V$ with respect to $x^{-1}$ and the $a_{ij}$ and $b_{ij}$ are smooth functions $N \rightarrow \mathbb{R}$. Let $A(p) = (a_{ij}(p))$ and let $B(p) = (b_{ij}(p))$. By the Gram-Schmidt construction both matrices are upper-triangular and invertible, so $C(p) = B(p)A(p)^{-1}$ is an upper-triangular matrix. We write $R^i_p = \sum_{j=1}^n c_{ij} Q^j_p$, and calculate:
\begin{equation*}
\sum_{i=m+1}^n v_i(p) R^i(p) = \sum_{i=m+1}^n v_i(p)(\sum_{j=1}^n c_{ij} Q^j_p) = \sum_{i=m+1}^n \sum_{j=m+1}^n c_{ij} v_i(p) Q^j_p
\end{equation*}
where the last inequality follows because $c_{ij} = 0$ whenever $j > i$, and $i$ is at least $m+1$. Therefore $\psi_1 \circ \psi_2^{-1}: U_1 \cap U_2 \times \mathbb{R}^{n-m} \rightarrow \mathbb{R}^m \times \mathbb{R}^{n-m}$ is the map
\begin{equation*}
(p,(v_{m+1},...,v_n)) \mapsto (p,(\sum_{j=m+1}^n c_{1j}(p) v_j,...,\sum_{j=m+1} c_{nj}(p) v_j))
\end{equation*}
By the smoothness of the $c_{ij}$, this a smooth function. 

Under these charts, $NM$ has a unique topology generated by $\psi^{-1}(V)$ for some $\psi$ corresponding to a chart $(\pi^{-1}(U),\psi)$ and $V \subset U \times \mathbb{R}^{m-n}$ with respect to which it is a differentiable manifold ("Smooth Manifold Chart Lemma" 1.35, Lee \textit{Introduction to Smooth Manifolds}).   Each $\psi$ is bijective, and is clearly a homeomorphism with respect to this topology, since the open sets are exactly the preimages of open sets in $U \times \mathbb{R}^n$. In addition, we can now confirm that $\pi: NM \rightarrow N$ is differentiable, since its coordinate representation with respect to a chart $(\pi^{-1}(U),\psi)$ on $NM$ and $(U,x)$ on $M$ is just $(x_1,...,x_n,a_1,...,a_{n-m}) \mapsto (x_1,...,x_n)$. Restricting $\psi$ to $\pi^{-1}(\lbrace p \rbrace)$ gives the linear transformation $\sum_{i=m+1}^{n+m} a_i Y^i_p \mapsto (a_{m+1},...,a_{m+n})$, which is a vector space isomorphism. This confirms Criterion (ii) in Gundmundsson Def. 4.1. This completes the proof. 

\newpage
\paragraph*{Extra 1: Calculations for Q.2}
\includegraphics[scale=.5]{calculations_1.pdf}
\end{document}