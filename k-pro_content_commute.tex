\documentclass[10pt,letter]{article}
\usepackage{amsmath,amssymb,graphicx,setspace,fullpage,breqn}
\onehalfspacing
\usepackage{fullpage}

\begin{document}
\textbf{I. Equivalent way of describing k-promotion}. Let $\text{D}_1: \text{Inc}^{[q]}(\lambda) \rightarrow \text{Inc}^{[q]\cup \bullet}(\lambda)$ be the map from increasing tableaux labeled with $1,...,q$ to increasing tableaux labeled with $1,...,q,\bullet$ that replaces all $1$s with $\bullet$s. Let $\text{Sw}_n$ be the operator that finds all short ribbons containing $\bullet$ and $n$, leaves trivial ribbons unchanged, and switches $n$ and $\bullet$ in nontrivial ribbons. Let $\text{Sw}$ be the operator that finds all instances of $\bullet$, determines the minimum integer $k \leq q$ such that $k$ labels a box directly right or below an instance of $\bullet$, and applies $\text{Sw}_k$. Let $SW$ be the operator that applies $Sw$ until all $\bullet$s have no boxes directly below or the right. Let $\Sigma$ be the operator that cyclically permutes the labels: $q \rightarrow q-1 \rightarrow ... \rightarrow 1 \rightarrow q$. 

In [1], k-promotion is described as $\Sigma \circ \text{Sw}_q \circ ... \circ \text{Sw}_1 \circ \text{D}_1$. We claim that this is equivalent to $\Sigma \circ \text{SW} \circ \text{D}_1$. 

COMPLETE

\textbf{II. k-promotion commutes with FC}. We can write $T \in \text{Inc}^{[q]\cup \bullet}(\lambda) \rightarrow \text{Inc}^{[q]\cup \bullet}(\lambda)$ as a set of pairs $\lbrace (n,(i,j)): (i,j) \in \lambda \rbrace$ with the appropriate order constraints on $n$. Let $Q(i) = n_i$, where $n_i$ is the $i^{th}$ element in $T$, ordered lowest to highest. Define $FC((n,(i,j)) = (Q^{-1}(n),(i,j))$, so that $FC(T) = \lbrace (Q^{-1}(n),(i,j)) \rbrace$. 

\end{document}